\documentclass[11pt]{article}
\usepackage{fullpage}

\begin{document}

%%%%%%%%%%%%%%%%%%%%%%%%%%%%%%%%%%%%%%%%%%%%%%%%%%%%%%%
% Title, author, and institution information
%%%%%%%%%%%%%%%%%%%%%%%%%%%%%%%%%%%%%%%%%%%%%%%%%%%%%%%
\title{\bf An open letter concerning \\
\vspace{5mm}{\em   A Simple Proof of the Riemann Hypothesis}}
\author{\large\em  Leon Oiler, John Grauss, Joe Lagrunge, John Dirishlay, and Joe Fouray }

\date{} % No date

%%%%%%%%%%%%%%%%%%%%%%%%%%%%%%%%%%%%%%%%%%%%%%%%%%%%%%%
\maketitle
%%%%%%%%%%%%%%%%%%%%%%%%%%%%%%%%%%%%%%%%%%%%%%%%%%%%%%%
\pagestyle{empty}

In our paper we provided a simple proof of the Riemann hypothesis,
along with an extension of this result illustrating that $P = NP$.
While we feel that this is an important result, and should be of
interest to the community, our paper did not received a warm welcome
by the reviewers.

One reviewer called our paper ``bipolar'', pointing to the fact that
Section 2 purported to offer a simple proof of the Riemann hypothesis,
while Section 3 constructed a counterexample.  In response to this
comment, we initially considered splitting these sections of the
manuscript into separate papers. However, we find it interesting that
none of the reviewers pointed out any flaw in either section when
viewed individually.  Based on this positive aspect of the reviews
(and despite the prejudicial \emph{ad hominem} attacks of the
reviewer questioning our mental capacities), we are unwavering in our
confidence that our results are correct.

Furthermore, one reviewer suggested that our proof that $P = NP$
should be omitted since it would not be of interest to the same
community.  However, we feel that this result is so tightly
connected to the Riemann hypothesis that presenting these proofs
separately would eliminate one of the chief contributions of this
paper. Consequently, we have elected to retain this proof even
though it is of secondary consequence to the main result of the
paper.

We are frankly perplexed as to why the reviewers struggled to identify
the significant contributions of our manuscript.  The editor-in-chief
rejecting our manuscript ensured us that the reviewers were experts in
number theory.  We (humbly) conclude that the writing in our original
manuscript must have been unclear and difficult to follow.  We have
made several edits to our manuscript to fix this flaw, including the
addition of Figure 1 to provide a graphical illustration of our proof
technique and a change in terminology to conform to the standards
already established in the literature (e.g., the title Section 4.2).

While we believe that this paper contains results that are interesting
to the mathematical community, we simply do not have the time or the
patience to continue working with the manuscript to resubmit it to
another journal.  We have therefore submitted it to \emph{Rejecta
  Mathematica} so we can move on the more important mathematical
problems that lay ahead of us.  We hope this community finds it to be
a valuable contribution.


\end{document}
