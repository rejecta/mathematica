\documentclass[11pt]{article}
\usepackage{geometry}                % See geometry.pdf to learn the layout options. There are lots.
\geometry{letterpaper}                   % ... or a4paper or a5paper or ... 
%\geometry{landscape}                % Activate for for rotated page geometry
%\usepackage[parfill]{parskip}    % Activate to begin paragraphs with an empty line rather than an indent
\usepackage{graphicx}
\usepackage{amssymb}
\usepackage{fullpage}
\usepackage{epstopdf}
\DeclareGraphicsRule{.tif}{png}{.png}{`convert #1 `dirname #1`/`basename #1 .tif`.png}

\topmargin 0.0in
\headheight 0.0in
\pagestyle{empty}


\begin{document}
%
%\begin{center}
%\LARGE{From the editors..}
%\end{center}
%

\begin{center}
\Large{\textit{An open letter concerning}}\\
\LARGE{Mass matrix transforms in qubit field theory}\\
\hfill\\
\large{Marni Sheppeard}
\end{center}

To whomever ...
\hfill\\

This small paper reports on the initial observation that Carl Brannen's mass operators are naturally expressed as discrete Fourier series, common in the theory of quantum computation. Our obsession with these simple matrices has generated a great deal of criticism. Lubos Motl, the string theorist, called us F--ing Crackpots on my blog, Arcadian Functor, and all attempts to have this paper endorsed for the preprint arxiv failed. Although it is to be recognized that the abundance of errors in much of my writing is regrettable, in my experience these errors are never corrected by the people who think that this work is trivial and wrong, because it would be beneath them to consider it seriously.

The difficulty here is that our motivation for studying these mass operators lies not in standard particle physics, nor in standard theories of gravity, neither of which have anything whatsoever to say about the rest masses of fundamental particles. Unfortunately, a basic idea in quantum field theory is that certain parameters, such as mass, vary continuously, in a very complex way, from singular raw values. An unwavering belief in this idea leads people to conclude that simple formulae for rest masses cannot exist. Of those knowledgeable people willing to consider that such formulae may exist, most appear to believe that one should make no attempt to publish papers about it until one has constructed a complete theory of quantum gravity. 

Since Carl Brannen, and many others, have now traveled a fair distance down this road, it seems that a rather impressive theory will actually exist before a single paper is published in a highly regarded peer reviewed journal. Fortunately, thanks to the Internet Age, a rapidly growing number of people are now working on this subject.

\end{document}  