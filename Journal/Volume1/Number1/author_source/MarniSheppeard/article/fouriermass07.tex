%Fourier Transform and Circulant Mass Matrices - M. D. Sheppeard 2007
%
%Copyright (c) 2001 The American Physical Society.
% This is a template for producing manuscripts for use with REVTEX 4.0
% For Phys. Rev. appearance, change preprint to twocolumn.
% Choose pra, prb, prc, prd, pre, prl, prstab, or rmp for journal
% Add 'draft' option to mark overfull boxes with black boxes
% Add 'showpacs' option to make PACS codes appear
% Add 'showkeys' option to make keywords appear
%----------------------------------------------------------------
\documentclass[aps,prl,onecolumn,showpacs,address,11pt]{revtex4}
%\documentclass{article}
\usepackage{amsmath}
\usepackage{amssymb}

\pagestyle{empty}

\begin{document}

\title{Mass matrix transforms in qubit field theory}

\author{M. D. Sheppeard}
\affiliation{Department of Physics and Astronomy \\ University of
Canterbury, Christchurch, New Zealand}

\begin{abstract}
Circulant mass matrices for triples of charged and neutral leptons
have been studied in the context of qubit quantum field theory.
This note describes the discrete Fourier transform behind such
matrices, and discusses a category theoretic interpretation of
these operators.
\end{abstract}

\pacs{03.67.-a, 03.67.Lx, 04.60.-m}


\maketitle
\thispagestyle{empty}

%----------------------------------------------------------------
\section{Introduction}

Using a measurement algebra approach to QFT, Brannen
\cite{Brannen} recently recovered the Koide \cite{Koide1} formula
\begin{equation} (\sqrt{m_{e}} + \sqrt{m_{\mu}} + \sqrt{m_{\tau}})^{2} = \frac{3}{2}
(m_{e} + m_{\mu} + m_{\tau}) \end{equation} for charged lepton
masses in the form of a $3 \times 3$ circulant complex matrix,
whose eigenvalues squared give the lepton masses to experimental
precision. This analysis was extended to a set of three neutrinos,
and the mass ratio predictions agree with preliminary neutrino
oscillation data.

Here it is observed that the discrete Fourier transform
\cite{AdrovBook} provides a further interpretation of the mass
matrices, both as a duality between operators and eigenvalues and
also as a link to the theory of quantum computation
\cite{NielsenChuang}.

It is expected that other triples of Standard Model particles,
namely baryons and mesons, will also be associated with $3 \times
3$ matrix operators of the same kind in accord with their preon
structure \cite{Brannen} and the association of spatial directions
to the number of particle generations, given by the three
primitive idempotents of the measurement algebra.

%----------------------------------------------------------------
\section{Fourier transforms and mass matrices}

A {\em circulant} matrix is built from its first row by adding
cyclic permutations. In particular, a $3 \times 3$ circulant takes
the form \begin{equation} \left( \begin{array}{ccc} A & B & C \\ C & A & B \\
B & C & A \end{array} \right) \end{equation} where $A$, $B$ and
$C$ will be complex numbers. Note that any such circulant is a
combination of the three permutations $(123)$, $(231)$ and
$(312)$. For real eigenvalues $\lambda_k$ it is essential that $A$
be real and $C = \overline{B}$. Thus a mass matrix \cite{Brannen}
takes the form
\begin{equation}\label{BrannenMat} C = \eta
\left( \begin{array}{ccc} 1 & r e^{i \theta}
& r e^{- i \theta} \\ r e^{- i \theta} & 1 & r e^{i \theta} \\ r
e^{i \theta} & r e^{- i \theta} & 1 \end{array} \right)
\end{equation} for real $\eta$, $r$ and $\theta$. In terms of
these parameters, the eigenvalues are given by
\[ \lambda_{k} = \eta (1 + 2r \textrm{cos}(\theta + \frac{2 \pi k}{3}))
\] The Koide formula (1) follows when $r^2 = \frac{1}{2}$ and this
choice may be applied also to the neutrino matrix.

In the $n \times n$ case, the discrete Fourier transform
\cite{AdrovBook}\cite{NielsenChuang} interchanges the set of
eigenvalues $\lambda_{k}$ (assumed distinct) and matrix entries
$A_1 , A_2, A_3 , \cdots , A_n$ via
\begin{eqnarray} \lambda_k = \sum_{j} e^{\frac{2 \pi i jk}{n}}
A_{j} \\ \nonumber A_j = \frac{1}{n} \sum_{k} e^{- \frac{2 \pi i
jk}{n}} \lambda_{k} \end{eqnarray} Viewing the eigenvalues as a
diagonal matrix, the transform interchanges the bases of
projection operators and cyclic permutations. For real eigenvalues
$(m_1 , m_2 , m_3)$ with $m_i = \lambda_{i}^{2}$ in the above, and
letting $\omega = e^{\frac{2 \pi i}{3}}$, the transform takes the
diagonal matrix to the circulant {\small
\[ \left( \begin{array}{ccc} m_1 + m_2 + m_3
& m_1 \omega + m_2 \omega^{2} + m_3 & m_1 \omega^{2} + m_2 \omega
+ m_3 \\ m_1 \omega^{2} + m_2 \omega
+ m_3 & m_1 + m_2 + m_3 & m_1 \omega + m_2 \omega^{2} + m_3 \\
m_1 \omega + m_2 \omega^{2} + m_3 & m_1 \omega^{2} + m_2 \omega +
m_3 & m_1 + m_2 + m_3
\end{array} \right) \] } which must be the square of
(\ref{BrannenMat}) since the square of a circulant is a circulant.
Thus a choice of scale is specified by $\eta = \frac{1}{3} (m_1 +
m_2 + m_3)$.

%Products and Quantum Torus
A $3 \times 3$ matrix is viewed as a function on the discrete
torus $\mathbb{Z}_{3} \times \mathbb{Z}_{3}$, which has a quantum
description and a convolution product for matrices
\cite{AdrovBook}. Letting $D_{ij} = \delta_{ij} \omega^{i}$ there is a Weyl rule \[ D \circ (312) = \omega (312)
\circ D \] where the phase $\frac{2 \pi}{3}$ is proportional to
$\hbar^{-1}$. This associates Planck's constant with a hierarchy
$\mathbb{N}$ determined by the size of the matrix, but the
continuum limit is obtained via $\hbar \rightarrow \infty$ rather
than $\hbar \rightarrow 0$.

If masses are to be thought of as quantum numbers, then why are
their values so awkward in comparison to, say, spin? For $2 \times
2$ circulants with entries $A$ and $B$, the eigenvectors are
$(1,1)$ and $(1,-1)$ with eigenvalues $(A + B)$ and $(A - B)$
respectively. For example, for the Pauli swap matrix $\sigma_{x}$,
with $A = 0$, the spin eigenvalues are $\pm 1$. Complexity in the
eigenvalue set only arises in dimension three or higher.

Degenerate eigenvalues $\frac{\lambda_{k}}{\eta} \in \{ 1 - r , 1
- r , 1 + 2r \}$ occur when $\theta = 0$ and all matrix entries
are real. Although this pattern does not describe the leptons, we
observe that it is the typical composition of masses for baryon
constituents. Since such mass operators arise in a preon model
that unifies particle structure, it is expected that all standard
model bound states and resonances may be arranged into mass
triples.

In quantum computation \cite{NielsenChuang} a Fourier transform is
also defined in this way, acting on a set of $n$ basis states. For example, an
$N$ qubit computation uses $n = 2^{N}$ basis states. The transform is
unitary and it may be built from unitary gates, namely the
Hadamard gate $H = \frac{1}{\sqrt{2}} (\sigma_{x} + \sigma_{z})$
and the series
\[ B_{k} = \left( \begin{array}{cc} 1 & 0 \\ 0 & e^{\frac{2 \pi i}{2^{k}}}
\end{array} \right) \] By analogy, a mass computation with $3^{N}$
basis states uses ternary digits, so the gates $B_{k}$ would be
replaced by gates \begin{equation} T_{k} = \left( \begin{array}{ccc} 1 & 0 & 0 \\
0 & e^{\frac{2 \pi i}{3^{k}}} & 0 \\ 0 & 0 & e^{\frac{4 \pi
i}{3^{k}}} \end{array} \right) \end{equation} which are also
unitary. In general, the Fourier operator entries $F_{ij}$ are given by $\omega^{ij}$, and the theory of {\em mutually unbiased bases} generalises the Pauli operator algebra in all prime power dimensions. 

A basic time evolution operator exists for each dimension $n$. Note, however, that unlike in conventional constructions, this local evolution is not in any way associated with an emergent cosmic clock, the latter being more closely related to the scale $\hbar$, given here by the matrix dimension. That is, this approach does not assume a globally defined time for a nonsensical universal
observer.

%----------------------------------------------------------------
\section{Discussion}

The mass matrices arise from a one dimensional discrete transform,
which itself involves commutative variables. However, it is seen
that phase space variables satisfy the Weyl algebra of the quantum
plane. Is there a noncommutative transform that extends this
analysis to nonclassical underlying spaces? This is relevant to
the question of extending the perturbative rest mass computations
\cite{Brannen} to nonperturbative regimes.

Kapranov \cite{KapranovFT} has recently considered path spaces
approximated by cubical paths, each of which is represented by a
noncommutative monomial in the spatial directions. In dimension $d
> 1$ a noncommutative Fourier transform relates measures on the
space of paths to functions of the noncommuting variables. The
basic idea is that a path integral is just a map from a
noncommutative ring to a suitable commutative subring. In this
way, particle masses \cite{Brannen} could arise as path integral
invariants.

Taking T-duality seriously, one also expects to deal with
nonassociativity. From a category theoretic point of view, both
noncommutative and nonassociative structures can be dealt with in
a unified framework. The cohomological element of interest here is
the parity cube axiom, which describes the now familiar pentagon
law on five of its faces. In a sufficiently lax algebraic setting,
such as for tetracategories, the sixth face may break this law,
providing the deformation parameter that turns a pentagon into a
hexagon representing the permutation group $S_3$
\cite{BataninComb}.

The generation count by primitive idempotents \cite{Brannen} is
confirmed by the string theoretic index theorem argument applied
to the Riemann moduli space of the six punctured sphere, which has
an orbifold Euler characteristic \cite{MulasePenk} of -6. The six
punctures are associated to the six faces of a cube via a dual
vertex, which is thickened to a sphere. Note that cohomological
integrals for such moduli spaces commonly appear in QFT
computations as multiple zeta values and polylogarithms.

%----------------------------------------------------------------
\begin{acknowledgments} For helpful discussions I thank Carl
Brannen, Michael Rios, Matti Pitkanen, Tony Smith and Louise
Riofrio. \end{acknowledgments}

\begin{thebibliography}{7}
\expandafter\ifx\csname natexlab\endcsname\relax\def\natexlab#1{#1}\fi
\expandafter\ifx\csname bibnamefont\endcsname\relax
 \def\bibnamefont#1{#1}\fi
\expandafter\ifx\csname bibfnamefont\endcsname\relax
 \def\bibfnamefont#1{#1}\fi
\expandafter\ifx\csname citenamefont\endcsname\relax
 \def\citenamefont#1{#1}\fi
\expandafter\ifx\csname url\endcsname\relax
 \def\url#1{\texttt{#1}}\fi
\expandafter\ifx\csname urlprefix\endcsname\relax\def\urlprefix{URL }\fi
\providecommand{\bibinfo}[2]{#2}
\providecommand{\eprint}[2][]{\url{#2}}

\bibitem[{\citenamefont{Brannen}()}]{Brannen}
\bibinfo{author}{\bibfnamefont{C.~A.} \bibnamefont{Brannen}},
 \bibinfo{note}{http://brannenworks.com/dmaa.pdf}.

\bibitem[{\citenamefont{Koide}(1982)}]{Koide1}
\bibinfo{author}{\bibfnamefont{Y.}~\bibnamefont{Koide}},
 \bibinfo{journal}{Lett. Nuovo Cim.} \textbf{\bibinfo{volume}{34}},
 \bibinfo{pages}{201} (\bibinfo{year}{1982}).

\bibitem[{\citenamefont{Aldrovandi}(2001)}]{AdrovBook}
\bibinfo{author}{\bibfnamefont{R.}~\bibnamefont{Aldrovandi}},
 \emph{\bibinfo{title}{Special matrices of mathematical physics}}
 (\bibinfo{publisher}{World Scientific}, \bibinfo{year}{2001}).

\bibitem[{\citenamefont{Nielsen and Chuang}(2000)}]{NielsenChuang}
\bibinfo{author}{\bibfnamefont{M.~A.} \bibnamefont{Nielsen}} \bibnamefont{and}
 \bibinfo{author}{\bibfnamefont{I.~L.} \bibnamefont{Chuang}},
 \emph{\bibinfo{title}{Quantum computation and quantum information}}
 (\bibinfo{publisher}{Cambridge}, \bibinfo{year}{2000}).

\bibitem[{\citenamefont{Kapranov}()}]{KapranovFT}
\bibinfo{author}{\bibfnamefont{M.}~\bibnamefont{Kapranov}},
 \bibinfo{note}{math.QA/0612411}.

\bibitem[{\citenamefont{Batanin}()}]{BataninComb}
\bibinfo{author}{\bibfnamefont{M.}~\bibnamefont{Batanin}},
 \bibinfo{note}{math.CT/0301221}.

\bibitem[{\citenamefont{Mulase and Penkava}()}]{MulasePenk}
\bibinfo{author}{\bibfnamefont{M.}~\bibnamefont{Mulase}} \bibnamefont{and}
 \bibinfo{author}{\bibfnamefont{M.}~\bibnamefont{Penkava}},
 \bibinfo{note}{math-ph/9811024}.

\end{thebibliography}

\end{document}
