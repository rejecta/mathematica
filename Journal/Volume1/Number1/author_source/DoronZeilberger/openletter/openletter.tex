\documentclass[11pt]{article}
\usepackage{geometry}                % See geometry.pdf to learn the layout options. There are lots.
\geometry{letterpaper}                   % ... or a4paper or a5paper or ... 
%\geometry{landscape}                % Activate for for rotated page geometry
%\usepackage[parfill]{parskip}    % Activate to begin paragraphs with an empty line rather than an indent
\usepackage{graphicx}
\usepackage{amssymb}
\usepackage{fullpage}
\usepackage{epstopdf}
\usepackage{hyperref}
\DeclareGraphicsRule{.tif}{png}{.png}{`convert #1 `dirname #1`/`basename #1 .tif`.png}

\topmargin 0.0in
\headheight 0.0in
\pagestyle{empty}


\begin{document}
\begin{center}
\Large{A note from the Rejecta Mathematica editorial board
regarding}\\ 
\LARGE{\emph{Automatic CounTilings} by Doron Zeilberger}
\end{center}

The following paper was submitted by Doron Zeilberger in response to an invitation to contribute a paper to the inaugural issue of Rejecta Mathematica. In lieu of a traditional open letter, we would like to refer the reader to Prof. Zeilberger's website:

\begin{center}
\href{http://www.math.rutgers.edu/~zeilberg/mamarim/mamarimhtml/tilings.html}{\textbf{http://www.math.rutgers.edu/$\sim$zeilberg/mamarim/mamarimhtml/tilings.html}}
\end{center}

As a brief summary, this paper considers the problem of computing the number of ways in which a $k \times n$ rectangle can be covered by a given set of tiles. The paper in fact describes a Maple program -- available at the website above -- which will tackle the problem for all $n$, given any $k$ and any set of allowable tiles. What was once a problem that would have to be tackled on a case-by-case basis is approached with a unified treatment that relies on a computer to discover the appropriate ``structure theorems.'' As Prof. Zeilberger claims on his website, ``what is so nice about it is that everything is done by machine: the combinatorics, the algebra, and the analysis.''

The rejection history of this paper is well-documented on Prof. Zeilberger's website. A primary source of dispute in the original referee review was whether Prof. Zeilberger's algorithm alone constituted a significant, original mathematical contribution. Prof. Zeilberger includes the original review, which he describes as ``narrow-minded and ignorant'', offers his own reply, and expresses his opinion about the ``flawed'' editorial policies of the rejecting journal. 

For full context, software code, and examples, the interested reader is invited to read both the paper and the website.

\hfill\\

\hfill Rejecta Mathematica Editors
\end{document}  