\documentclass[11pt]{article}
\usepackage{geometry}                % See geometry.pdf to learn the layout options. There are lots.
\geometry{letterpaper}                   % ... or a4paper or a5paper or ... 
%\geometry{landscape}                % Activate for for rotated page geometry
%\usepackage[parfill]{parskip}    % Activate to begin paragraphs with an empty line rather than an indent
\usepackage{fullpage}
\usepackage{graphicx}
\usepackage{amssymb}
\usepackage{epstopdf}
\DeclareGraphicsRule{.tif}{png}{.png}{`convert #1 `dirname #1`/`basename #1 .tif`.png}

\pagestyle{empty}



\begin{document}

\begin{center}
\LARGE{\emph{An open letter concerning}\\WInHD: Wavelet-based Inverse Halftoning via Deconvolution}\\
\hfill\\
\large{Ramesh Neelamani and Richard Baraniuk}
\end{center}

\hfill

\noindent {\bf Birth}: The niche problem of inverse halftoning error-diffused halftones
has been addressed by a number of solid researchers using several
practical and effective methods. However, due to the non-linearity of
the halftoning process and the complexities of the human visual system,
the methods proposed to date have been ad hoc.

At first glance, we thought that we had little chance of coming up with
even a mediocre solution to the nonlinear inverse halftoning problem. We
pursued lines of research from photon-limited imaging and Polya trees,
but those approaches lead nowhere.  One day, Rob Nowak found some
literature on an intriguing linear approximation to halftoning. We were
pleasantly surprised when a wavelet-thresholding based estimator based
on this linear approximation produced competitive results (not only in
terms of the workhorse mean-squared-error (MSE) metric but also in terms
of a standard visual quality metric). We called our algorithm
Wavelet-based Inverse Halftoning via Deconvolution (WInHD).

We thought that WInHD would be a ``slam-dunk" paper that would certainly
interest the image processing community, because in addition to
presenting competitive results near the state-of-the-art, our insights
also reduced the inverse halftoning problem to a well-understood
deconvolution problem. Furthermore, assuming that the linear
approximation was accurate and that the model noise was Gaussian, we
were able to derive and analyze bounds on WInHD's MSE performance as the
image resolution increased.

With high optimism, we submitted a paper to a top-tier image processing
journal.

\noindent {\bf Death}: But alas, our enthusiasm was deflated due to the following review
points, which we disagree with.
\begin{itemize}
\item The linear approximation was deemed questionable. Any claims about optimality were deemed to be overstated.
\item Our results were deemed to be visually inferior. The metrics used
used to evaluate our simulation results did not conform to the quality
of the images as perceived. We were urged to seek input from the experts
in the field and then publish the results of the survey.
\end{itemize}
The combination of lukewarm reviews and diverging author interests meant
that the paper had to be abandoned.

\noindent {\bf After-life}: With its publication in Rejecta Mathematica, we would like to honestly address some of the issues raised in our paper's day of reckoning.

We believe that while the reviewers raised several valid points, the
paper contained several contributions that would benefit the image
processing community.  Addressing the linear approximation point, we
agree that a linear approximation to the halftoning process is not
suitable for all purposes. However, our view is that the surprising
results obtained using such a model make our paper more, not less,
interesting.  We do concede that the optimality claims made in the paper
need to be taken with a this linear approximation in mind.  However, the
limitations of our analysis have been clearly stated in the paper (it
was termed as conditional optimality in the paper, but perhaps our
analysis required some bigger and bolder disclaimers).

On the visual quality issue, beauty indeed lies in the eyes of the
beholder!  Like a majority of image processing practitioners, we agree
that the MSE may be inadequate to measure the visual quality of an
image. However, in our paper, we employed all of the metrics that were
accessible in the literature (that is, we did not cherry-pick them) to
substantiate that our method provided ``superior visual" performance
(arguably a strong term to use, but certainly not obviously wrong).
Surveys can certainly be an effective approach to analyzing an image
processing result. But, while useful, conducting surveys for every image
processing paper borders on onerous. As an alternative, we published our
code so that our results were reproducible and so that our method could
be tested on anyone's images of choice.

The tussle about the visual quality improvement afforded by WInHD seems
to have no easy resolution in sight. However an even larger question
emerges. Is it really necessary for follow-on papers to always
significantly improve upon previous results?  Should a paper's
publishablity be so heavily reliant on the improved results that it
produces?  How about insights that may open some closed doors?



\end{document}  