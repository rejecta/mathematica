\documentclass[12pt]{article}
\usepackage{geometry}                % See geometry.pdf to learn the layout options. There are lots.
\geometry{letterpaper}                   % ... or a4paper or a5paper or ...
\usepackage{graphicx,url}
\usepackage{amssymb}
\usepackage{fullpage}
\usepackage{epstopdf}
\DeclareGraphicsRule{.tif}{png}{.png}{`convert #1 `dirname #1`/`basename #1 .tif`.png}

\usepackage[pagebackref=false,citecolor=black,colorlinks=true,linkcolor=black,urlcolor=black]{hyperref}

\usepackage[numbers,sort&compress]{natbib}

\pagestyle{empty}


\begin{document}

Welcome to the inaugural issue of Rejecta Mathematica! Thank you for
joining us for what we hope will be a unique and interesting
experiment. For those unfamiliar with our mission, Rejecta
Mathematica is an open access, online journal that publishes only
papers that have been rejected from peer-reviewed journals in the
mathematical sciences. In addition, every paper appearing in Rejecta
Mathematica includes an open letter from its authors discussing the
paper's original review process, disclosing any known flaws in the
paper, and stating the case for the paper's value to the community.

Since starting this endeavor, the questions we've been asked most
often are ``Why are you doing this?'' and ``Is it a joke?'' While we
are not above admitting that we have had a few good laughs in this
process, we hope that this issue will serve as definitive proof that
Rejecta Mathematica is not a joke. Despite the central role that
peer review (and even rejection) must play in the scientific
process~\cite{Fang05012008}, we believe there are several reasons
why this project can make a positive and valuable contribution to
the mathematical sciences research community.

First, there is ample evidence that in the traditional review
process, significant (even Nobel prize-winning) research is
occasionally overlooked and groundbreaking work is sometimes
actively
shunned~\cite{natureCoping,barber1961rss,Cam::1996::Have-referees}.
Perhaps this is most dramatically illustrated in the fact that at
least ``36 future Nobel Laureates encountered resistance on [the]
part of scientific journal editors or referees to manuscripts that
dealt with discoveries that on [a] later date would assure them the
Nobel Prize"~\cite{nodelrejected}. While it would be presumptuous
for us to assume that we can spot significant work that others may
have missed, we can provide a venue to introduce rejected work to
the community and increase the chances that its value will be
appreciated sooner rather than later.

Second, there is also evidence that a research community can derive
value from a centralized repository of rejected papers, even when
(and perhaps especially when) the results are either incorrect or
not significant enough to warrant consideration for a major
international prize. Rejecta Mathematica can benefit authors looking
for feedback on their work, wanting to warn the community against
false starts (i.e., the classic ``null results'' that never see the
light of day, only to be repeated by others)~\cite{stallings,jnrbm}, or
wanting to illuminate the occasional vagaries of the peer review
process to enhance accountability and scientific
integrity~\cite{philica}. Our journal can also benefit readers who
want access to ``minor results'' that may be useful but not
publishable in isolation. Indeed, Rejecta Mathematica has existed in
folklore for many years as a fictitious place to send papers that
were never to see the light of day, and the concept of a formal
repository for rejected papers hoping to be discovered and revived
(called Rejuvenatable Mathematics) has also been
proposed~\cite{Mag::1997::Theorems-that}.

While such a project as Rejecta Mathematica would have been
impracticable in the pre-internet age, the flood of resources
available today begs another oft-posed question: ``Why do we need a
new journal? Isn't this what a preprint server (like the arXiv), a
blog, or a personal website is for?'' We believe that a central
collection of articles that have been selected for their potential
interest to the community will increase their visibility beyond what
could be achieved through a general preprint server or personal
website. We also believe that the commentary and advocacy by the
authors will increase the value of the papers beyond what would
exist from the appearance of the paper alone. Finally, we believe
that the availability of thoughtful technical discussion (via
Rejecta Mathematica ``correspondences'' following up on previously published articles) has the potential to generate
more valuable interaction than the immediate commentary generally
available on a blog. There is no doubt, however, that blogs and
online archives can also play a significant role in advocating for
rejected papers.

Finally, we would be remiss not to mention that being researchers
ourselves, at some level we simply wanted to conduct an experiment.
What started as a fleeting idea around the lunch table (discussing
one of our own rejected papers) turned into the type of inquiry that
fuels even the most serious of studies: if we build Rejecta
Mathematica and ask for papers, what will happen? Will we get any
papers, and if so, will they all be the delusional output of
mathematical cranks? (This has been a common conjecture.)

Other questions concern our editorial policies. Should we simply
publish every article we receive, and if not, how should we evaluate
the submissions? After careful consideration, we have settled on an
editorial process that includes no technical peer review (hence our
slogan ``Caveat Emptor''). Rather, we will rely on the technical
review provided by the journal from which the paper was originally
rejected and focus instead on selecting papers based on their
apparent potential interest to researchers in the mathematical
sciences. Admittedly, and perhaps necessarily in a journal of this
scope, the concept of ``potential interest'' encompasses a broad set
of loosely defined criteria. Ultimately, we will try to choose
papers that allow some opportunity for learning. For example, we do
not see much value to the community in publishing papers that were
rejected solely for their incomprehensibility.

The open letter plays a major part in our decision process, as we
view its role in a Rejecta Mathematica article as being at least as
important as the technical content of the research paper. The open
letters are where the authors can both tell the history of the
paper and convey the lessons learned from the rejection.
Undoubtedly, many open letters will provide a frank commentary on
the peer-review process. Some may even be controversial. At the very
least, they should help others benefit from the (technical and
nontechnical) mistakes of their peers. To address the original
question, there have indeed been papers rejected from Rejecta
Mathematica.

We are delighted to say that the content of this first issue runs
the gamut of genres included in our mission: minor or traditionally
unpublishable results, non-traditional ideas and proof techniques,
misunderstood genius, results based on questionable assumptions, and
controversial papers and open letters. We are also pleased that the
papers span several areas of the mathematical sciences, including
pure mathematics, applied mathematics, theoretical physics, and
engineering. We hope that you enjoy the issue with as much good
humor and intellectual stimulation as we have encountered in putting
it together.  We welcome feedback, future submissions, and support
for the Rejecta Mathematica mission through our website:
\url{math.rejecta.org}.

\hfill
\begin{flushright}
\emph{Michael  Wakin --- Christopher  Rozell --- Mark  Davenport --- Jason  Laska}\\
{\footnotesize \textsf{editors@rejecta.org}\\}
\end{flushright}

\small
\bibliographystyle{IEEEtran}
\bibliography{coverletter}

\end{document}
