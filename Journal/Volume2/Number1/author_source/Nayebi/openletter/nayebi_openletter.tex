\documentclass[11pt]{article}
\usepackage[top=1in,right=1in,left=1in,bottom=1in]{geometry}
\usepackage{fullpage}

\pagestyle{empty}

\begin{document}

%%%%%%%%%%%%%%%%%%%%%%%%%%%%%%%%%%%%%%%%%%%%%%%%%%%%%%%
% Title, author, and institution information
%%%%%%%%%%%%%%%%%%%%%%%%%%%%%%%%%%%%%%%%%%%%%%%%%%%%%%%
\title{\bf An open letter concerning \\
\vspace{5mm}{\em  On the distribution of Carmichael numbers}}
\author{\large\em  Aran Nayebi }

\date{} % No date

%%%%%%%%%%%%%%%%%%%%%%%%%%%%%%%%%%%%%%%%%%%%%%%%%%%%%%%
\maketitle
%%%%%%%%%%%%%%%%%%%%%%%%%%%%%%%%%%%%%%%%%%%%%%%%%%%%%%%
\thispagestyle{empty}

In my paper, entitled ``On the distribution of Carmichael numbers'', I investigate the distribution of Carmichael numbers. The importance of Carmichael numbers is that they test the limits of Fermat's primality test, which ultimately led mathematicians to formulate more effective primality tests in the twentieth century. There have been two important conjectures regarding the distribution of these numbers up to sufficiently large bounds, one made by Paul Erd\H{o}s in 1956 and a subsequent sharpening of this conjecture by Carl Pomerance in 1981. However, neither of these conjectures are well-supported by the Carmichael number counts famously performed by Richard Pinch up to $10^{21}$. The inaccuracies of these two aforementioned conjectures are understandable, since not too much is known about Carmichael numbers. In fact, after a century of investigation regarding these numbers, it was only a decade ago that the infinitude of Carmichael numbers was proven! In this paper, I present two conjectures (which sharpen Erd\H{o}s' and Pomerance's conjectures) regarding the distribution of Carmichael numbers that fit proven bounds, are roughly supported by Pinch's data (as well as data from other papers and resources), that closely model the true distribution of Carmichael numbers, and are supported by many theorems and conjectures put forth by renowned mathematicians such as Alford, Erd\H{o}s, Galway, Granville, Harman, Pomerance, Wagstaff, Selfridge, and Szymiczek. The reader may wonder why \emph{two} conjectures are presented. The reason is that due to the lack of information regarding Carmichael numbers and their distribution, both conjectures are viable to their own merit.

Unfortunately, although I feel that the results in this paper are important and would satisfy the interests of the mathematical community, the paper was rejected by three journals.

The first journal the paper was submitted to was \emph{Mathematics of Computation}. The referee stated that ``the paper deals with interesting topics and might be generally appropriate for Math. Comp. �However, the paper is written very poorly and it needs a lot of work before it can be properly considered.'' Thus, I humbly took the advice of the referee, and I spent the better part of two months revising the paper rigorously with a colleague of mine. I made the paper more readable, the notation more recognizable, and I added six data tables from various cited sources (some of the data I collected myself), all in support of my conjecture. Similarly, through this revision process, we disproved many of my conjectures and theorems, and we sharpened and strengthened many of my proofs. However, the only conjecture that we were \emph{unable to disprove} was my conjecture regarding Carmichael numbers.  Furthermore, I discussed my paper with several mathematicians who are known for their work on Carmichael numbers and pseudoprimes (which are a superset of Carmichael numbers), all of whom agreed with the majority of my ideas. I also requested feedback from a mathematician who had not published any papers in this field, who stated: ``I read through your paper on pseudoprimes, and while the subject is not my area of expertise, it is clear that you are familiar with the mathematical literature and are making a serious contribution.''

After revising the paper thoroughly, I then submitted the paper to \emph{The American Mathematical Monthly}. Although they were unable to find any mistakes (both mathematical and style-wise) and this time the paper received a good editorial review, the paper was rejected because ``the Monthly tries to publish expositions of mathematics that are accessible to a broad mathematical audience. The material in your paper is rather technical, and we feel that many Monthly readers will find it forbidding. We will therefore not be able to accept it for publication. These are difficult decisions. The Monthly receives a large number of submissions each year, and we are able to publish only a small fraction of them.''

I could not help but be amused by this rejection notice; however, I was somewhat flustered. Carmichael numbers are important in number theory because of their rarity (there are only 20138200 Carmichael numbers up to $10^{21}$), and their existence demonstrates the ineffectiveness of the Fermat primality test. Furthermore, the fact that not too much is known about these numbers after almost a century of research, means that more work about them should be considered for inclusion within mathematical literature. Also, my paper is not forbidding as there are tables which present my assertions in non-verbiage form and these tables are even explained in detail. The notation is also entirely readable and widely-recognized.

As a final straw, I sent the paper to Carl Pomerance, in the hopes of a more extensive and in-depth peer review. At the time, Conjecture 1.0.4 (the second conjecture) had not been included in the manuscript; only Corollary 1.0.3 (the first conjecture) was presented as the main result. Although my correspondence with him was brief (parts of which I include in my paper), his advice was helpful. Pomerance's arguments in support of his conjecture compelled me to propose a second conjecture that was a refinement to his original 1981 one, mainly by utilizing finer estimates for the distribution of smooth numbers (a practice which he stated had not yet been done before). This conjecture, which later became Conjecture 1.0.4, gave extremely accurate counts for $C(x)$, the number of Carmichael numbers up to $x$, at least for smaller bounds (although asymptotically it is the same result as Pomerance's).

With these adjustments made, I submitted my manuscript to \emph{Experimental Mathematics} as it is ``a journal devoted to the experimental aspects of mathematics research.'' Unfortunately, two months later, they rejected the submission on the grounds that ``the two conjectures presented by the author can each be substantially simplified by using known (or easily derived) asymptotics for the constituent parts....The first conjecture is extremely unlikely to be true, if only for the reason that it postulates an asymptotic formula for the number of Carmichael numbers up to $x$, while no other conjecture makes such a strong statement....Also, in the second conjecture, the author claims to be including more explicit secondary terms, but the $(1+o(1))$ factor just washes them out anyway. In short, the statements would need to be substantially simplified and polished to make this paper worth publishing in a strong journal such as EM.'' I agree with the referee that the statements would have to be simplified; a task which I had completed prior to submission, even going so far as to provide numerical estimates for the various constants used in the statement of Corollary 1.0.2. However, my points of contention with the referee are that the first conjecture cannot simply be disregarded as untrue due to the strength of its assertions (and in fact the numerical evidence compiled in my paper demonstrates its viability) and that the second conjecture must include secondary terms in it so that the discrepancies pointed out by Pinch will not occur.

%Essentially Pomerance stated that ``in the numerical computation of the function $h(x)$, you've
%noted that it is not making much progress towards the conjectural limit of 1.  In my paper from 1981, I suggest some secondary terms may be present.  So, say in my %conjecture, one replaces ``$\log\log\log x$'' with ``$\log\log\log x + \log\log\log\log x$''. It is the same conjecture, since the two are asymptotic, and if you %define $k(x)$ analogously, then (I only computed starting
%at $10^{15}$), the values decrease steadily from 1.571 to 1.524 at $x=10^{21}$, and so the Pinch phenomenon is banished.
%Further, if the very next term is added in, it is $\log_4(x)/\log_3(x)$, and one defines $j(x)$ as with $k(x)$ and $h(x)$, it also decreases steadily from 1.398 to %1.343, and it is not hard to believe in either case that the limit is 1.  Both sequences have monotone decreasing second differences, modeling a function which is %decreasing and concave up, which suggests a horizontal asymptote, which may as well be 1. The heuristic from which my conjecture was derived involved the distribution %of smooth numbers, and then I used analytic approximations for this to get the conjecture.  If instead one used the Dickman rho function, one might get better %agreement with the actual count of Carmichaels, but I don't believe anyone reworked my heuristic using this function.  There are even finer approximations to this %distribution, as in the work of Hildebrand and Tenenbaum.''

If anything, the second conjecture appears to be more plausible than the first; however, both conjectures provide different and intriguing insights into the distribution of Carmichael numbers. The first conjecture asserts that an asymptotic formula for $C(x)$ easily follows based on the computation of numerical constants. The second conjecture indicates to us that if secondary terms exist, then the properties of smooth number counting functions \emph{must} be examined further in order to effectively prove an equality for $C(x)$.

Frankly, submitting the paper to another peer-reviewed journal and waiting a few months to a year for a referee look over a paper which has already been examined by several mathematicians of the same expertise (if not more) is a waste of time. I have submitted my paper to \emph{Rejecta Mathematica} in the hopes of advancing mathematics and the investigation of pseudoprimes and their variants.



\end{document} 