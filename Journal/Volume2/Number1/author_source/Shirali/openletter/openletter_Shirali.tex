\documentclass[11pt]{article}
\usepackage[top=1in,right=1in,left=1in,bottom=1in]{geometry}
\pagestyle{empty}
%\usepackage{fullpage}
%\topmargin 0.0in
%\headheight 0.0in
%\pagestyle{empty}

\begin{document}

%%%%%%%%%%%%%%%%%%%%%%%%%%%%%%%%%%%%%%%%%%%%%%%%%%%%%%%
% Title, author, and institution information
%%%%%%%%%%%%%%%%%%%%%%%%%%%%%%%%%%%%%%%%%%%%%%%%%%%%%%%
\title{\bf An open letter concerning \\
\vspace{5mm}{\em   Extended real number system in measure theory
}}
\author{\large\em Satish Shirali}

\date{} % No date

%%%%%%%%%%%%%%%%%%%%%%%%%%%%%%%%%%%%%%%%%%%%%%%%%%%%%%%
\maketitle
\thispagestyle{empty}
%%%%%%%%%%%%%%%%%%%%%%%%%%%%%%%%%%%%%%%%%%%%%%%%%%%%%%%

\noindent This article was originally submitted under the title
``Extended real number system''. Among the reasons given for its
rejection was that there was undue focus on methods involved, such
as consideration of separate cases, and that too little had been
said about the relationship between what the review called ``the
proposed system'' and other number systems that include infinity.\

However, {\em no new system is being proposed} in the article and
the very first sentence of the second paragraph includes the
phrase ``is usually defined as'' in order to prevent any
misinterpretation in this regard. But evidently to no avail.\

Referring to the rule that infinity times zero should be zero as
the rule ``you propose,'' the review agreed with my observation that
this does not work out well in many situations. However, the rule
in question is quite standard in the extended real number system
used for measure and integration, and nowhere does the article
suggest that all conceivable systems are being studied under one
roof.\

\vspace{4mm} The focus on multiplicity of cases has been argued
for in the body of the article: Since multiplication in extended
reals is defined by separating positive reals, negative reals,
zero, positive infinity and negative infinity, verifying
associativity alone requires an enormous number of dissimilar
cases to be considered. The author feels that a construction
procedure for the extended reals, involving only a manageable
number of dissimilar cases---and this is what the article is
mainly about, though not exclusively---is worth having on record.\

Unfortunately, this is the only issue that is emphasized in the
abstract. Within the article however, it has been pointed out that
there is legitimate cause to question the consistency of a system
having the usual properties which are assumed to hold for extended
reals, and furthermore, that doubts arising on this score have
been laid at rest in the rest of the discussion. The author feels
that this is another feature that makes the exposition worth
placing on record.\

\vspace{4mm} Besides a change in the title so as to include the
phrase ``in Measure Theory,'' there is an amendment in the abstract
that reads ``For the extended reals as used in measure theory
(product $0\cdot\infty$ is 0)'' in place of ``As an alternative.''
Also, the word ``dissimilar'' has been inserted.

\end{document}
