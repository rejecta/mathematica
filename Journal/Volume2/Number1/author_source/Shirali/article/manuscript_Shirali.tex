% Generated by GrindEQ Word-to-LaTeX 2007
% LaTeX/AMS-LaTeX

\documentclass{article}
\usepackage[top=1in,right=1in,left=1in,bottom=1in]{geometry}
\pagestyle{empty}
%\usepackage{fullpage}
%\topmargin 0.0in
%\headheight 0.0in
%\pagestyle{empty}

%%% remove comment delimiter ('%') and specify encoding parameter if required,
%%% see TeX documentation for additional info (cp1252-Western,cp1251-Cyrillic)
\usepackage[cp1252]{inputenc}

%%% remove comment delimiter ('%') and select language if required
%\usepackage[english,spanish]{babel}

\usepackage{amssymb}
\usepackage{amsmath}
\usepackage[dvips]{graphicx}
%%% remove comment delimiter ('%') and specify parameters if required
%\usepackage[dvips]{graphics}

\begin{document}


\title{Extended real number system in measure theory}

\author{Satish Shirali\thanks{Email: satishshirali@usa.net, Address: House No.899, Sector 21, Panchkula, Haryana 134116, India}}

\date{}

\maketitle
\thispagestyle{empty}


\begin{abstract}
\noindent The extended real number system is usually defined by
appending two new elements and stating rules of addition and
multiplication for them. The associative and distributive laws are
then supposed to be verified case by case; however, the number of
cases to be verified is well over sixty. For the extended reals as
used in measure theory (product $0\cdot\infty$ is 0), we offer a
construction through equivalence classes, in which the number of
dissimilar cases does not exceed five at any
stage.
\end{abstract}





\vspace{1cm}





\noindent In any proof that requires consideration of separate
cases, usually the number of cases is small and it is quite clear
that all of them have been taken into account. However, when the
number of cases is large, it may not be so clear that none have
been left out. For example, verifying the associative law for a
binary operation described in tabular form on a set of $n$
elements is impractical when $n$ exceeds 4. Another such
instance is that of the associative and distributive laws in the
extended real number system.

The extended real number system $\tilde{{\bf {\mathbb R}}}$ is
usually defined as the union of ${\bf {\mathbb R}}$ with two
elements, written $\infty$ and $-\infty $, and endowed with the
structure described in (a)---(h) below, in addition to that
already available on its subset ${\bf {\mathbb R}}$:\\

(a) $-\infty <x<\infty$ for every $x\in\bf {\mathbb R}$;

(b) $x+\infty=\infty +x=\infty$ and $x+(-\infty)=(-\infty)+x=-\infty$ for every $x\in {\bf {\mathbb R}}$;

(c) $\infty+\infty=\infty$ and $(-\infty)+(-\infty)=-\infty$;

(d) $\infty\cdot\infty=(-\infty)\cdot(-\infty)=\infty$ and
 $(-\infty)\cdot\infty=\infty\cdot(-\infty)=-\infty$;


(e) $-(-\infty)=\infty$ and $-(\infty)=-\infty$;

(f) $x\cdot\infty=\infty\cdot x=\infty$ and $x\cdot(-\infty)=(-\infty)\cdot x=-\infty$

\indent \indent \indent for every positive $x\in\bf
{\mathbb R}$;

(g) $x\cdot(\infty)=(\infty)\cdot x=-\infty$ and $x\cdot(-\infty)=(-\infty)\cdot x=\infty$

\indent \indent \indent for every negative \textit{x} $\in$ ${\bf
{\mathbb R}}$;

(h) $(\pm\infty)\cdot 0=0\cdot(\pm\infty)=0$.\\



\noindent In contexts other than measure and integration, one may
wish to omit (h) and take the products occurring in it as
``undefined''. For instance, it is omitted in \cite[p.12] {rudin1}
with the consequence on p.314 that the Lebesgue integral of the
identically zero function on ${\bf {\mathbb R}}$ is left undefined
by (53), considering that (49) requires the function to be written
as zero times the characteristic function of ${\bf {\mathbb R}}$.
However, the same author includes (h) in \cite[p.19] {rudin2}.

Without (e), there would seem to be no basis for the common
practice of regarding $x-(-\infty)$ as meaning $x+\infty$ and
$x-(\infty)$ as meaning $x+(-\infty)$. We have therefore chosen to
state it explicitly although most authors prefer not to.


It is immediate from the properties (a)---(h) that addition and multiplication
in $\tilde{{\bf {\mathbb R}}}$ are commutative. However, the
associative law of addition and the distributive law, which
continue to be valid under the restriction that $\infty$ and
$-\infty $ do not both appear in any of the sums involved, are
supposed to be verified case by case. The associativity of
multiplication can be verified, again case by case, to be valid
without restriction.

The multiplication described by (a)---(h) is, in effect, a
binary operation on a set of five elements, namely,
positive real, negative real, 0, $\infty$  and $-\infty$ . In
checking associativity therefore, one would have to consider 125
triplets $(x,y,z)$; however, 27 of these involve only real numbers
and need not be checked. The remaining 98 can be reduced to 68 by
taking advantage of the obvious commutativity, but this is still an
uncomfortably large number of cases to handle.

Consequently, in any effort at a case-by-case verification of the
associative and distributive laws in the
extended real number system, it would be a legitimate concern
whether all cases have actually been taken into account or not.

We have avoided adding the requirement that $\frac{x}{\infty}$ = 0 for $x
\in\bf {\mathbb R}$ so as to keep clear of the consequence
that

\begin{center} 0 = 0$\cdot\infty$ =
$\frac{1}{\infty}\cdot\infty$ but $\frac{1\cdot\infty}{\infty}$ is
undefined.\end{center}


\noindent However, this observation raises a second concern, namely, whether (a)---(h) already contain a contradiction, even without this requirement.\\



With a view to addressing both concerns, we outline a method of
``constructing" $\tilde{{\bf {\mathbb R}}}$ from ${\bf {\mathbb
R}}$, in which we describe $\infty$ and $-\infty $ set
theoretically in terms of ${\bf {\mathbb R}}$ rather than pull them out
of the sky, and moreover, the associative and distributive laws
become transparent with just two cases each. The definitions of addition and multiplication in $\tilde{{\bf
{\mathbb R}}}$ undoubtedly call for separate cases to be
considered, but it is transparent that none are left out.\\


We begin by describing what the objects $\infty$ and
$-\infty $ are.

Let $\infty$ denote the class of real sequences ``diverging to $\infty$" in the usual sense (no circularity involved in this) and $-\infty $ denote the obvious analogous class. Furthermore, for each \textit{x} $\in$ ${\bf {\mathbb R}}$, let $\left[\kern-0.15em\left[x\right]\kern-0.15em\right]$ denote the class consisting of a single sequence, namely, the constant sequence with each term equal to \textit{x}. Set $\left[\kern-0.15em\left[{\bf {\mathbb R}}\right]\kern-0.15em\right]$ = \{$\left[\kern-0.15em\left[x\right]\kern-0.15em\right]$ : \textit{x} $\in$ ${\bf {\mathbb R}}$\} and $\tilde{{\bf {\mathbb R}}}$ = $\left[\kern-0.15em\left[{\bf {\mathbb R}}\right]\kern-0.15em\right]$$\cup $\{$-\infty $,$\infty$\}. Then each element of $\tilde{{\bf {\mathbb R}}}$ is a class of sequences and the classes are disjoint.

For $\alpha$,$\beta\in\tilde{{\bf {\mathbb R}}}$, define $\alpha<\beta$ to mean: for any sequences \{$a_{n}$\} $\in\alpha$ and \{$b_{n}$\} $\in\beta$, the inequality $a_{n}<b_{n}$ holds for all sufficiently large \textit{n}. Then it is easy verify that

\begin{equation} \label{ZEqnNum961558} \left[\kern-0.15em\left[x\right]\kern-0.15em\right]<\left[\kern-0.15em\left[y\right]\kern-0.15em\right]\Leftrightarrow x<y{\rm \; \; if\; }x,y\in {\bf {\mathbb R}} \end{equation}

\noindent Also, $-\infty<\alpha<\infty$ for all $\alpha\in\left[\kern-0.15em\left[{\bf {\mathbb R}}\right]\kern-0.15em\right]$, and $-\infty $ $<$ $\infty$. Thus (a) holds with \textit{x} replaced by $\left[\kern-0.15em\left[x\right]\kern-0.15em\right]$.

Suppose $\alpha$,$\beta\in\tilde{{\bf {\mathbb R}}}$, and $\alpha\neq-\infty\neq\beta$. It is easy to see that the following four cases are exhaustive:
\[{\rm (i)}\alpha ,\beta \in \left[\kern-0.15em\left[{\bf {\mathbb R}}\right]\kern-0.15em\right]{\rm \; \;  (ii)}\alpha \in \left[\kern-0.15em\left[{\bf {\mathbb R}}\right]\kern-0.15em\right]{\rm \; }{\rm a}{\rm n}{\rm d}{\rm \; }\beta =\infty {\rm \; \;  (iii)}\alpha =\infty {\rm \; }{\rm a}{\rm n}{\rm d}{\rm \; }\beta \in \left[\kern-0.15em\left[{\mathbb R}\right]\kern-0.15em\right]{\rm \; \;(}{\rm i}{\rm v}{\rm )}\alpha ,\beta =\infty .\]

\noindent It is equally straightforward to see in each of the four
cases that, for any sequences \{$a_{n}$\} $\in\alpha$ and
\{$b_{n}$\} $\in\beta$, the related sequence \{$a_{n}+b_{n}
$\} belongs to a unique $\gamma\in\tilde{{\bf {\mathbb
R}}}$. Therefore we may define $\alpha+\beta$ to be this unique
$\gamma\in\tilde{{\bf {\mathbb R}}}$.

In the course of arguing for the unique $\gamma$, it is also seen that

\[\left[\kern-0.15em\left[x\right]\kern-0.15em\right]+\infty =\infty +\left[\kern-0.15em\left[x\right]\kern-0.15em\right]=\infty {\rm \; \; if\; \; }x\in {\bf {\mathbb R}}{\rm ,}\]
\[\infty +\infty =\infty \]

\begin{flushleft}
\noindent and
\begin{equation} \label{ZEqnNum487027} \left[\kern-0.15em\left[x\right]\kern-0.15em\right]+\left[\kern-0.15em\left[y\right]\kern-0.15em\right]=\left[\kern-0.15em\left[x+y\right]\kern-0.15em\right]{\rm \; if\; }x,y\in {\bf {\mathbb R}} \end{equation}
\end{flushleft}

One can proceed analogously when $\alpha$,$\beta\in\tilde{{\bf {\mathbb R}}}$, and $\alpha\neq\infty\neq\beta$. All this establishes that $\alpha+\beta$ is uniquely
defined except when one of them is $\infty$ and the other is
$-\infty$ and that addition satisfies (c) as well as the
properties claimed for it in (b), but with $x$ replaced by
$\left[\kern-0.15em\left[x\right]\kern-0.15em\right]$.\\

Now consider $\alpha$,$\beta\in\tilde{{\bf {\mathbb R}}}$. When $\alpha$,$\beta\in\left[\kern-0.15em\left[{\bf {\mathbb R}}\right]\kern-0.15em\right]$, any sequences \{$a_{n} $\} $\in$ $\alpha$ and \{$b_{n} $\} $\in$ $\beta$ must be constant sequences and it is immediate that the related (constant) sequence \{$a_{n} b_{n} $\} belongs to a unique $\delta$ $\in$ $\left[\kern-0.15em\left[{\bf {\mathbb R}}\right]\kern-0.15em\right]$ $\subseteq$ $\tilde{{\bf {\mathbb R}}}$. When $\alpha$ = $\infty$, the following five cases for $\beta$ are exhaustive:

\[\left[\kern-0.15em\left[0\right]\kern-0.15em\right]<\beta \in \left[\kern-0.15em\left[{\bf {\mathbb R}}\right]\kern-0.15em\right],{\rm \; \; \; \; }\left[\kern-0.15em\left[0\right]\kern-0.15em\right]>\beta \in \left[\kern-0.15em\left[{\bf {\mathbb R}}\right]\kern-0.15em\right],{\rm \; \; \; \; }\left[\kern-0.15em\left[0\right]\kern-0.15em\right]=\beta ,{\rm \; \; \; \; }\beta =\infty ,{\rm \; \; \; \; }\beta =-\infty .\]

\noindent In each of the five cases, it is easy to arrive at the
conclusion that: for any sequences \{$a_{n} $\} $\in\alpha$ and
\{$b_{n} $\} $\in\beta$, the related sequence \{$a_{n} b_{n}
$\} belongs to a unique $\delta\in\tilde{{\bf {\mathbb
R}}}$. We note that it is essential here for $\beta$ =
$\left[\kern-0.15em\left[0\right]\kern-0.15em\right]$ to consist
of only the constant sequence \{0,0,\dots\}. Similarly when
$\alpha$ = $-\infty$. In view of the commutativity of
multiplication in ${\bf {\mathbb R}}$, the same conclusion can be
drawn for the ten cases when $\beta$ = $\pm\infty $. (It is
inessential to the argument that the actual number of distinct
cases is not 20 but 16, because the four cases when $\alpha$ =
$\pm\infty $ and $\beta$ = $\pm\infty $ will occur twice among
the 20.) Thus all cases have been covered and the aforementioned
conclusion holds for all $\alpha$,$\beta\in\tilde{{\bf
{\mathbb R}}}$. Therefore we may define $\alpha\beta$ to be this
unique $\delta\in\tilde{{\bf {\mathbb R}}}$.

In the course of arguing for the unique $\delta$, it is also seen that multiplication satisfies

\begin{equation} \label{ZEqnNum938796} \left[\kern-0.15em\left[x\right]\kern-0.15em\right]\left[\kern-0.15em\left[y\right]\kern-0.15em\right]=\left[\kern-0.15em\left[xy\right]\kern-0.15em\right]{\rm \; if\; }x,y\in {\bf {\mathbb R}} \end{equation}

\noindent as well as (d),(h), and that it further satisfies
(f),(g), with $x$ replaced by
$\left[\kern-0.15em\left[x\right]\kern-0.15em\right]$.\\

For any $\alpha\in\tilde{{\bf {\mathbb R}}}$, there is a
unique element $-\alpha\in\tilde{{\bf {\mathbb R}}}$ such
that

\[-\left[\kern-0.15em\left[\alpha\right]\kern-0.15em\right]=\left[\kern-0.15em\left[-\alpha\right]\kern-0.15em\right]{\rm \; if\; \; }\alpha\in {\bf {\mathbb R}}\]

\noindent and $-(-\infty)=\infty$, $-(\infty)=-\infty$. Indeed, $-\alpha$ is the unique element of $\tilde{{\bf {\mathbb R}}}$ such that whenever \{$a_{n}$\} $\in\alpha$, the sequence \{$-a_{n}$\} belongs to the class $-\alpha$. This proves (e).

In view of \eqref{ZEqnNum961558}, \eqref{ZEqnNum487027} and \eqref{ZEqnNum938796}, the bijection $x\rightarrow\left[\kern-0.15em\left[x\right]\kern-0.15em\right]$ is an isomorphism of ordered fields. Thus the subset $\left[\kern-0.15em\left[{\bf {\mathbb R}}\right]\kern-0.15em\right]$ of $\tilde{{\bf {\mathbb R}}}$ is an isomorphic image of ${\bf {\mathbb R}}$.
\\

Having completed the construction of a system satisfying (a)---(h)
and containing an isomorphic image of ${\bf {\mathbb R}}$, we now
turn our attention to the associative and distributive laws.

Suppose that either none among $\alpha$,$\beta$,$\gamma$ is
$\infty$ or that none is $-\infty$. Let \{$a_{n}$\} $\in\alpha$, \{$b_{n}$\} $\in\beta$ and \{$c_{n}$\} $\in\gamma$. Then $(\alpha+\beta)+\gamma$ is the unique class in
$\tilde{{\bf {\mathbb R}}}$ containing the sequence \{$(a_{n}+b_{n})+c_{n}$\}, while $\alpha+(\beta+\gamma)$ is the
unique class in $\tilde{{\bf {\mathbb R}}}$ containing the
sequence \{$a_{n}+(b_{n}+c_{n})$\}. By the associativity of
addition in ${\bf {\mathbb R}}$, it follows that the classes are
the same. Thus the equality

\[(\alpha+\beta)+\gamma =\alpha+(\beta+\gamma)\]

\noindent holds provided that

\[{\rm either\; none\; among\; }\alpha {\rm ,\; }\beta {\rm ,\; }\gamma {\rm \; is\; }\infty \]
\[{\rm or\; none\; among\; }\alpha {\rm ,\; }\beta {\rm ,\; }\gamma {\rm \; is\; }-\infty .\]

\noindent Similarly, the equality

\[\alpha(\beta+\gamma)=\alpha\beta+\alpha\gamma\]

\noindent holds provided that

\[{\rm either\; none\; among\; \; }\beta {\rm ,\; }\gamma ,{\rm \; }\alpha \beta {\rm ,\; }\alpha \gamma {\rm \; is\; }\infty \]
\[{\rm or\; none\; among\; \; }\beta {\rm ,\; }\gamma ,{\rm \; }\alpha \beta {\rm ,\; }\alpha \gamma {\rm \; is\; }-\infty .\]

\noindent In fact, both sides of the equality are the unique class
containing the sequence \{$a_{n}(b_{n}+c_{n})$\}, where
\{$a_{n}$\} $\in\alpha$, \{$b_{n}$\} $\in\beta$ and
\{$c_{n}$\} $\in\gamma$. It is left to the reader to formulate
the corresponding statement regarding $(\alpha\beta)\gamma$ =
$\alpha(\beta\gamma)$, which is valid without any restrictions
on $\alpha$,$\beta$,$\gamma$.

\paragraph{}\

We conclude with two remarks.\\

\indent 1. If
$\left[\kern-0.15em\left[x\right]\kern-0.15em\right]$ were
enlarged to include all real sequences converging to $x$,
then the properties (a)---(g) would follow in the same manner as
above, but $\pm\infty\cdot 0$ and $0\cdot (\pm\infty)$
would remain undefined.\\
\indent 2. If in the construction of $\mathbb{R}$ by Dedekind cuts
as in \cite [pp.17-21]{rudin1} or \cite [pp.47-52]{SV}, one
includes the empty set and $\mathbb{Q}$ as cuts, then one gets
$\tilde{{\bf {\mathbb R}}}$, with the empty set serving as
$-\infty$ and $\mathbb{Q}$ as $\infty$. The additional effort
involved in checking this is minimal. However, the sum
$-\infty+\infty$ needs to be specifically excluded in the general
definition of sum, because otherwise it works out to be $-\infty$.


\begin{thebibliography}{2}



\bibitem{rudin1} Rudin, W., Principles of Mathematical Analysis, 3rd ed., McGraw-Hill, New York,
1976

\bibitem{rudin2} Rudin, W., Real and Complex Analysis, McGraw-Hill, New York, 1966

\bibitem{SV} Shirali, S. and Vasudeva, H.L., Mathematical
Analysis, Alpha Science Publishers, Oxford, 2006.







\end{thebibliography}



\end{document}
