\documentclass[11pt]{article}
\usepackage[top=1in,right=1in,left=1in,bottom=1in]{geometry}
\pagestyle{empty}
%\usepackage{fullpage}
%\topmargin 0.0in
%\headheight 0.0in
%\pagestyle{empty}

\begin{document}

%%%%%%%%%%%%%%%%%%%%%%%%%%%%%%%%%%%%%%%%%%%%%%%%%%%%%%%
% Title, author, and institution information
%%%%%%%%%%%%%%%%%%%%%%%%%%%%%%%%%%%%%%%%%%%%%%%%%%%%%%%
\title{\bf An open letter concerning \\
\vspace{5mm}{\em   Explanation of low Hurst exponent for Riemann zeta zeros   }}
\author{\large\em O. Shanker }

\date{} % No date

%%%%%%%%%%%%%%%%%%%%%%%%%%%%%%%%%%%%%%%%%%%%%%%%%%%%%%%
\maketitle
\thispagestyle{empty}
%%%%%%%%%%%%%%%%%%%%%%%%%%%%%%%%%%%%%%%%%%%%%%%%%%%%%%%

In 2006 a striking result was published
(Generalised Zeta Functions and Self-Similarity of Zero Distributions, J.
Phys. A 39(2006), 13983-13997) about the
statistics of the zeros of the Riemann zeta function.
The paper (by the current author) applied rescaled range analysis,
and found that the
zeros  exhibited an unusably low Hurst Exponent. While that paper
discussed some possible
explanations, no clear reason for the low Hurst Exponent emerged.
This was particularly
interesting because the most obvious explanation, that the
differences of the zeros have a very large anti-correlation,
 would be very unusual indeed.

Recently the author came up with empirical evidence for another
less radical explanation,
and submitted the explanation for publication, and it was rejected.
The main reason for the rejection appears to be the limited
content.  While the content is empirical
and limited to a single point, the author
nevertheless feels that
the result should be published, not as an important new result, but
because it provides a plausible
explanation for the original findings. The referee also commented that
the author did not mention that the
distribution of the fluctuating part of the zero counting function has
been known to be Gaussian distributed, and did not explain what the
Hurst exponent was.
The author agrees that the paper may
benefit by having more detail and references. However, the original article had
detailed references and discussion  about the literature
concerning the distribution of the zeros of the Riemann zeta function, so
the reader may rely on the original
paper for details and references. The author is not aware of any
errors in the paper, and has not made any changes in the paper.

The referee also mentioned that there may be many distributions
other than the Gaussian that would produce the same results.
The author agrees: the key point of the paper is that the rescaled
range analysis does not necessarily imply a long-range anti-correlation
in the differences of the zeros, and to give an indication of other
possible, less radical explanations.

\end{document}
