\documentclass[11pt]{article}
\usepackage[top=1in,right=1in,left=1in,bottom=1in]{geometry}
\pagestyle{empty}
%\usepackage{fullpage}
%\topmargin 0.0in
%\headheight 0.0in
%\pagestyle{empty}

\begin{document}

%%%%%%%%%%%%%%%%%%%%%%%%%%%%%%%%%%%%%%%%%%%%%%%%%%%%%%%
% Title, author, and institution information
%%%%%%%%%%%%%%%%%%%%%%%%%%%%%%%%%%%%%%%%%%%%%%%%%%%%%%%
\title{\bf An open letter concerning \\
\vspace{5mm}{\em   The problematic nature of G\"{o}del's theorem }}
\author{\large\em Hermann Bauer and Christoph Bauer }

\date{} % No date

%%%%%%%%%%%%%%%%%%%%%%%%%%%%%%%%%%%%%%%%%%%%%%%%%%%%%%%
\maketitle
\thispagestyle{empty}
%%%%%%%%%%%%%%%%%%%%%%%%%%%%%%%%%%%%%%%%%%%%%%%%%%%%%%%

Our paper ``The problematic nature of G\"{o}del's theorem'' was rejected by MLQ (Mathematical Logic Quarterly). The managing editor (admittedly) has not read it, and no reviews were provided.  His argument in favor of rejection was essentially that ({\em a}) ``G\"{o}del's results and techniques of proof are well-acknowledged by the scientific community \ldots since a lot of versed logicians have given detailed and well-understandable presentations and modifications of these results and proofs in a lot of frequently read textbooks.'' Additionally ({\em b}) the editor claims that in general texts like ours, which try to disprove well-accepted results, ``\ldots finally demonstrate only a lack of understanding by the authors.  It cannot be the task of editors and referees to disprove all these `disprovers'. In some cases, their errors are obvious, sometimes it takes a considerable amount of time to point out them. Who would spend this?''

In answer to ({\em a}) we would like to note that to our knowledge all secondary authors have directly assumed and sometimes even increased the problems of G\"{o}del's original work we raise in our paper. As to ({\em b}) we would like to state that if indeed all or most such critics hitherto demonstrate only a lack of understanding by the authors, it may be improbable, but not impossible, that our criticism is nonetheless valid. As we are convinced that it actually is we would further like to point out that {\em mathematical truth is not a question of probability.}  We recognize that many editors and referees, already over-burdened, must necessarily perform a certain degree of ``triage'' as papers are submitted---but does this not constitute a significant hole in the peer-review system?  Is it possible to publish a mathematical paper which challenges the accepted orthodoxy?

We think that our paper could and should create a useful and necessary discussion about G\"{o}del's theorem although, and in fact precisely because, it is a well established and unquestioned part in mathematical literature.


\end{document}
