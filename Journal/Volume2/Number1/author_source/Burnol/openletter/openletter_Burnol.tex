\documentclass[11pt]{article}
\usepackage[top=1in,right=1in,left=1in,bottom=1in]{geometry}
%\pagestyle{empty}
\usepackage{fullpage}
%\topmargin 0.0in
%\headheight 0.0in
\pagestyle{empty}

\begin{document}

%%%%%%%%%%%%%%%%%%%%%%%%%%%%%%%%%%%%%%%%%%%%%%%%%%%%%%%
% Title, author, and institution information
%%%%%%%%%%%%%%%%%%%%%%%%%%%%%%%%%%%%%%%%%%%%%%%%%%%%%%%
\title{\bf An open letter concerning \\
\vspace{5mm}{\em Scattering,
determinants, hyperfunctions in
relation to $\frac{\Gamma(1-s)}{\Gamma(s)}$}}
\author{\large\em Jean-Fran\c{c}ois Burnol}

\date{} % No date

%%%%%%%%%%%%%%%%%%%%%%%%%%%%%%%%%%%%%%%%%%%%%%%%%%%%%%%
\maketitle
\thispagestyle{empty}
%%%%%%%%%%%%%%%%%%%%%%%%%%%%%%%%%%%%%%%%%%%%%%%%%%%%%%%

I wrote this paper in 2006, and
submitted it to a journal specializing in integral equations
and operator theory. After circa 14 months I received a
report which I reproduce in full here (I allow myself to
correct the spelling of a mathematician's name cited in
the report):

\smallskip

\emph{``In spite of desperate efforts, the referee has failed to
  understand what the paper is about. Apparently it does not
  have a definite goal but consists of miscellaneous remarks
  to the papers by de~Branges and Rovnyak. It is practically
  impossible to distinguish original results in this
  jumble. Actually, the text does not look as a mathematical
  article
  but rather as some notes for personal use.\\
  In the referee's opinion, the paper should be rewritten
  according to conventional rules and its volume should be
  divided by the factor 5-10.  The author should try to
  formulate the results which he considers to be new.''
}

\smallskip


Let me explain why I consider  the publication of
the paper important. First of all the referee's report only serves to
demonstrate that the referee did not read the manuscript. I
tried to point this out to the editor in chief, to no avail:

\smallskip
``
\emph{
Dear Professor Burnol,\\
I read all your letters to us. I am not changing my mind!
Your paper is not accepted for publication. This decision is final and the
discussions about this paper this time I consider finished.
Sincerely, XXX
}
''
\smallskip

I think this illustrates nicely how dysfunctional the
peer-review process may be, at times. Regarding the paper
itself, it is well structured, and its goal was to prove new
mathematical theorems (!), a goal which was achieved (!). I
corrected a typo in 2008 (there was a superfluous imaginary
$i$ in some equations, see the footnote on page 1), this is
the only change to the 2006 version.

The referee asked me to divide the ``volume'' by between
five and ten, a request which at that time particularly
infuriated me. In fact, a more acceptable comment would have
been to point out that the paper contained material for
between 3 and 5 reasonably sized quasi-independent
publications (of reasonable, but obviously not earth-shaking
interest!), but I wanted to make a common exposition with in
particular a common introduction. What would be the point of
repeating 5 times the same introduction? An introduction is
made necessary by the fact that my perspective is unique and
links together a priori disjoint topics, the reader needs
some help in entering this framework.

Another difficulty is that in 2008, during a stay at Institut des Hautes \'{E}tudes Scientifiques (IHES), I
made very significant advances
(establishing links with domains apparently completely
unrelated, and which moreover have been of great interest
for the last thirty years to large communities of researchers), on
which I have had opportunities to give lectures at IHES, at
the European Conference of Mathematics (ECM) at Amsterdam, and at a workshop at the Independent
University of Moscow (Conference Zeta functions II). I have
circulated a hand-written manuscript of about 80 pages, and
prior to publishing this novel material in peer-reviewed journals,
I need to make my earlier work available to the
mathematical community.

I did sufficiently serious and dedicated work on this in 2006 resulting in a
paper of about 65 pages. It would be all too easy, and far
more beneficial to my career, to instead divide the paper
into at least 3 publications, but I just don't see the
point. If one is not sufficiently committed to mathematics
to place great importance on the form one gives to one's own
contributions, if one is ready to obey
arbitrary diktats, if all that matters is adding lines of
publications to a CV, then one practices a job and not a
passion and one does not care about his/her legacy, one
lives amidst superficial illusions and pleasures.

This paper will be necessary reading to get a full understanding of my
earlier as well as of my future works.



\end{document}
ines of
publications to a c.v., then one practices a job and not a
passion and one does not care about his/her legacy, one
lives amidst superficial illusions and pleasures. This paper
will be necessary reading to get a full und 